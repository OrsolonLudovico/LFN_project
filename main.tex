\documentclass{article}
\usepackage{graphicx}
\PassOptionsToPackage{hyphens}{url}\usepackage{hyperref}
\usepackage{titling}
\setlength{\droptitle}{-13em}
\usepackage{enumitem}
\setlist[description]{leftmargin = 0mm}
\usepackage{amsmath}
\usepackage{amsfonts}
\usepackage{float}

\title{An analysis of the \textit{friendship paradox} on different types of social networks}
\author{Ippolito Lavorati, Ludovico Orsolon, Patrizia Stefani}
\date{November 2024 - January 2025}


\begin{document}
\maketitle
\section{Motivations}
In recent decades, social media have become increasingly popular, transforming the way people connect and interact; examining these platforms from a graph network perspective can thus provide valuable insights into the social dynamics they enclose.\\
One particularly interesting phenomenon that we decided to explore is the \textit{friendship paradox}, a concept formulated by sociologist Scott L. Feld, which suggests that, "on average, an individual's friends have more friends than the individual does." \cite{friendship_paradox}.
This analysis will investigate how each considered platform reflects this paradox, aiming to understand the extent to which the phenomenon appears across these networks.

\subsection{Datasets}
We’ve chosen three graphs, each one representing the friendship in a different social network:
\begin{itemize}
    \item Youtube: the dataset at \url{https://snap.stanford.edu/data/com-Youtube.html} contains information about the networks of followers on Youtube communities 
    \item Linkedin: the dataset at \url{https://networkrepository.com/soc-linkedin.php} contains the user-to-user connections
    \item Facebook: the dataset at \url{https://networkrepository.com/socfb-wosn-friends.php} contains a network of users' friendships
\end{itemize}
We think that these three social networks are different enough in nature to have different characteristics between each other; for example, we expect that a 'popular user' on youtube will have many more 'friends' than the average one, while 'popular users' on Facebook will have fewer friends in proportion due to the fact that often unpopular users are still friends with a fair number of people (such as friends and family in real life).\\
Furthermore, since the datasets utilized are huge, we designed a smaller graph to debug our program and verify our measures (Figure~\ref{fig:testgraph}). We called this graph \textit{TestGraph}.
\begin{figure}[H]
    \centering
    \includegraphics[width=0.8\textwidth]{TestGraph.png}
    \caption{\textit{TestGraph}}
    \label{fig:testgraph}
\end{figure}
We created a random graph for each dataset in order to better verify the statistical analysis. The random graphs are constructed with the same number of nodes and edges of the real graph utilized using the \textit{gnm\_random\_graph} function from the \textit{NetworkX} library.\\
Generating a single random graph for each social network is certainly not optimal; we would need many random graphs to better estimate random features. We chose not to do so for the sake of computational time: computing even a single measure on our machines was a long process, so we decided to employ our time trying to understand results rather than computing tons of measures.\\
The graphs employed, as well as the respective random graphs are characterized by the following properties:
\begin{table}[ht]
\centering
\begin{tabular}{|c|c|c|c|c|}
\hline
& TestGraph & Facebook & Linkedin & Youtube \\ \hline
Nodes & $22$ & $63\,731$ & $6\,726\,011$ & $1\,134\,890$\\ \hline
Edges & $34$ & $817\,090$ & $19\,360\,690$ & $2\,987\,624$ \\ \hline
Avg Degree & 3.09 & 25.64 &  5.76 & 5.26 \\ \hline 
\end{tabular}
\caption{Some graphs' properties}
\end{table} 
%----------------put a list of every graph and how many nodes/ edges it has, maybe average degree or smth.... Also for random graphs

\section{Methods}
The program responsible for the analysis is written in \textit{Python} using the \textit{NetworkX} library for graphs, the \textit{tqdm} library to keep track of long processes through progress bars, \textit{pyplot} from \textit{matplotlib} to visualize information, \textit{pandas} to deal with csv files. The files containing edge information, downloaded from the datasets, will serve as input to our programs.Some statistical analysis were also performed in R
The program was developed collaboratively on GitHub\cite{Github_repo} and all computations provided in this paper can be found in one of the functions provided on it.

\subsection{The fship\_score}
We decided to encapsulate the concept of the friendship score in a measure that we called \textit{fship\_score}, computed for each node. 
The \textit{fship\_score} is defined as:
\\\[
\textit{\textbf{fship\_score(u)}} = \deg(u) \cdot \left( \frac{\sum_{v \in N(u)} \deg(v)}{\deg(u)} \right)^{-1} =
\frac{\deg(u)^{2}}{\sum_{v \in N(u)} \deg(v)}
\]\\
That is, the degree of the considered node divided by the average degree of its neighbors. Intuitively if the score is less or equal to 1 the friendship paradox is true for the node, otherwise we consider it an outlier.


%\section{Intended experiments}
%We will conduct several experiments focused on analyzing graph structures representing friendships or follows aiming at identifying key patterns among different types of social networks and also their differences.
%We will compute other node and graph features and use them to understand if some of those measures correlate with the 'popularity' of a node.
%We will compare our results with \textit{random graphs} to check whether the paradox applies even in networks that do not represent friendships. \\
%To understand if the \textit{fship score} is correlated with node features and to check how similar it is to the random graph, we will employ statistical tests. The exact tests to be used are still to be determined.

\section{Experiments}
\subsection{Verifying the \textit{friendship paradox}}
First of all we need to check if the \textit{friendship paradox} is present in our datasets: below a table with the proportion of outliers is reported (\textit{number of nodes with $fship\_score > 1$ $/$ $total\_number\_of\_nodes$}), this table contains results from one of our R scripts. 
\begin{table}[ht]
\centering
\begin{tabular}{|c|c|c|c|c|}
\hline
& TestGraph & Facebook & Linkedin & Youtube \\ \hline
Real & 0.2777 & 0.0865 & 0.2927 & 0.0325\\ \hline
Random & 0.3636 & 0.4086 & 0.3556 & 0.3544 \\ \hline
\end{tabular}
\caption{Proportion of outliers}
\end{table} 
\\

We can immediately see that the \textit{frienship paradox} is true for both real and random graphs but we can also see that it is more accentuated on the real graphs than on the random ones: this suggests that the \textit{friendship paradox} is a characteristic of real "social network" graphs even though it can be also present in random graphs to some degree. Furthermore, we can see that the paradox is more accentuated on Youtube and less on Linkedin as expected. From the data, the random graphs seem to have a different proportion of outliers than the real graphs (H1), to verify this hypothesis we performed the z-test on the values in the above table using [real graph - random corresponding graph] pairs.
\[z = \frac{p_1 - p_2}{\sqrt{p_{comb} (1 - p_{comb}) \left( \frac{1}{n_1} + \frac{1}{n_2} \right)}}\]
where:
\[n_i = \textit{number of nodes of i}\]
\[k_i = \textit{number of nodes with fship\_score $>$ 1}\]
\[p_i = \frac{k_i}{n_i}\]
\[p_{comb} = \frac{k_1 + k_2}{n_1 + n_2}\]


These are the resulting p-values: \\
\begin{table}[ht]
\centering
\begin{tabular}{|c|c|c|c|c|}
\hline
& TestGraph & Facebook & Linkedin & Youtube \\ \hline
Real vs Random & 0.5 & 0 & 0 & 0\\ \hline
\end{tabular}
\caption{p-values Real vs Random z-test (computed by our R function)}
\end{table} 
\\
The p-values are all zero (reject \(H_0\)) except for the artificial graph suggesting that the proportion of outliers are different between the random and the real graphs, this means that the \textit{fship\_score} is probably not due to randomness. We also compared datasets against each other to see if there are common proportions between them, but the p-values are zero for all combinations (facebook-Linkedin, Linkedin-Youtube and Youtube-Facebook) so we conclude that such correlation doesn't exists.

\subsection{\textit{Friendship paradox} correlations}
The relationship between the friendship paradox and various network metrics reveals intriguing insights into the structural dynamics of social networks. To delve into this phenomenon, we analyzed three critical metrics: \textit{degree}, \textit{PageRank}, and \textit{clustering coefficient}, each offering a unique perspective on how the paradox manifests within different regions of a graph.
\begin{itemize}
    \item Degree: represents the number of direct connections a node has, is central to the friendship paradox. By its very definition, the paradox arises from comparing a node’s degree with the average degree of its neighbors. Consequently, examining the degree metric allows us to identify the fundamental structural conditions that drive this phenomenon. Higher-degree nodes often exhibit more pronounced friendship paradox scores, as they are typically connected to similarly high-degree neighbors, skewing the averages upward.
    \item PageRank: provides a broader measure of a node’s importance within the network. Unlike degree, which considers only direct connections, PageRank evaluates both the quantity and quality of a node’s connections, factoring in the influence of neighbors. In our analysis, we observed a strong positive relationship between PageRank and the friendship paradox score. This suggests that influential nodes, those with high PageRank, are disproportionately affected by the paradox, likely due to their role as hubs within the network.
    \item Clustering coefficient: adds another layer of complexity to our understanding. This metric measures the extent to which a node’s neighbors are interconnected, forming tightly knit clusters. Interestingly, our findings indicate a weak negative correlation between clustering coefficient and the friendship paradox score. Nodes embedded in highly clustered regions of the network appear to experience a dampened paradox effect, potentially because the interconnected nature of their neighbors reduces variability in degrees.
\end{itemize}
To quantify these relationships, we employed a combination of statistical methods. \textit{Linear regression} provided a clear view of how the friendship paradox score changes with respect to each metric, while \textit{Pearson} and \textit{Spearman} correlations offered complementary perspectives on the strength and nature of these associations. 

\section{Possible improvements}
With more powerful computational machines and more time there are a few ideas that could be developed to expand on this paper results:
\begin{itemize}
    \item Generating many random graphs for each social network would provide a better approximation of the randomly-generated features and also, generating other types of random graphs (Chung-Lu model for example), could highlight new correlations we may have missed.
    \item An obvious improvement could be simply using more data, for example finding new graphs from other social networks, we could also check if in different graphs obtained from the same social network we find significative differences on \textit{fship\_score}.
\end{itemize}

\section{Machine specifications}
\begin{itemize}
    \item 8 cores CPU, 4.8 Ghz sustained clock while boosting
    \item 32 GB of 3600 Mhz 
    \item SSD $\approx$ 3000 MB/s in read and write
\end{itemize}
\section*{Contributions}
We worked together, we belive that the the difference in contribution amount between all of us was negligable, hence we report a contibution of 1/3 for each member.
\begin{description}[font=\normalfont\itshape]
    \item[Lavorati Ippolito:] Helped writing the report, helped finding the datasets, researched and presented ideas, Statistical tests, pyhton programming
    \item[Orsolon Ludovico:] Helped writing the report, gave the main idea about the friendship paradox, formulated of the hypothesis, researched about experiments, python programming, statistical test idea and interpretation
    \item[Stefani Patrizia:] Helped writing the report, helped finding the datasets, researched on the methods to implement the experiments and relative python functions, implemented some methods for the computation of graph statistics
\end{description}
\begin{thebibliography}{9}
\bibitem{friendship_paradox}
Scott L. Feld (1991) \emph{Why Your Friends Have More Friends
than You Do}, The University of Chicago.
\bibitem{Github_repo}
The program can be found on GitHub at \url{https://github.com/OrsolonLudovico/LFN_project.git}
\end{thebibliography}
\end{document}

