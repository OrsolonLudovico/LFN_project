\documentclass{article}
\usepackage{graphicx}
\PassOptionsToPackage{hyphens}{url}\usepackage{hyperref}
\usepackage{titling}
\setlength{\droptitle}{-13em}
\usepackage{enumitem}
\setlist[description]{leftmargin = 0mm}
\usepackage{amsmath}
\usepackage{amsfonts}

\title{An analysis of the \textit{friendship paradox} on different type of social networks}
\author{Ippolito Lavorati, Ludovico Orsolon, Patrizia Stefani}
\date{November 2024}


\begin{document}
\maketitle
\section{Motivations}
In recent decades, social media have become increasingly popular, transforming the way people connect and interact; examining these platforms from a graph network perspective can thus provide valuable insights into the social dynamics they enclose.\\
One particularly interesting phenomenon that we decided to explore is the \textit{friendship paradox}, a concept formulated by sociologist Scott L. Feld, which suggests that, "on average, an individual's friends have more friends than the individual does." \cite{friendship_paradox}.
This analysis will investigate how each considered platform reflects this paradox, aiming to understand the extent to which the phenomenon appears across these networks.

\subsection{Datasets}
We’ve chosen three graphs, each one representing the friendship in a different social network:
\begin{itemize}
    \item Youtube: the dataset at \url{https://snap.stanford.edu/data/com-Youtube.html} contains information about the networks of followers on Youtube communities 
    \item Linkedin: the dataset at \url{https://networkrepository.com/soc-linkedin.php} contains the user-to-user connections
    \item Facebook: the dataset at \url{https://networkrepository.com/socfb-wosn-friends.php} contains a network of users' friendships
\end{itemize}
We think that these three social networks are different enough in nature to have different characteristics between each other; for example, we expect that a 'popular user' on youtube will have many more 'friends' than the average one, while 'popular users' on Facebook will have fewer friends in proportion due to the fact that often unpopular users are still friends with a fair number of people (such as friends and family in real life).

\section{Methods}
The program responsible for the analysis will be written in \textit{Python} using mostly the \textit{NetworkX} library, which contains different utilities to deal with graphs. The files containing edge information, downloaded from the datasets, will serve as input to our programs. \\
The exact computation of some features in graphs of this size is unfeasible; hence we will use approximation for features such as the betwenness centrality; the program will be developed collaboratively on GitHub.

\subsection{The fship\_score}
We decided to encapsulate the concept of the friendship score in a measure that we called \textit{fship\_score}, computed for each node. 
The \textit{fship\_score} is defined as:
\\\[
\textit{\textbf{fship\_score(u)}} = \deg(u) \cdot \left( \frac{\sum_{v \in N(u)} \deg(v)}{\deg(u)} \right)^{-1} =
\frac{\deg(u)^{2}}{\sum_{v \in N(u)} \deg(v)}
\]\\
That is, the degree of the considered node divided by the average degree of its neighbors. Intuitevely if the score is less or equal to 1 the friendship paradox is true for the node, otherwise we consider it an outlier.


\section{Intended experiments}
We will conduct several experiments focused on analyzing graph structures representing friendships or follows aiming at identifying key patterns among different types of social networks and also their differences.
We will compute other node and graph features and use them to understand if some of those measures correlate with the 'popularity' of a node.
We will compare our results with \textit{random graphs} to check whether the paradox applies even in networks that do not represent friendships. \\
To understand if the \textit{fship score} is correlated with node features and to check how similar it is to the random graph, we will employ statistical tests. The exact tests to be used are still to be determined.

\section{What changed from deadline 1?}
We understood better what we can and should do to analyze the graphs and also prepared a program to compute and group the results 
together. We defined matters such as what exactly means measuring the friendship paradox in a more formal way and resized our scope.

\section{Machine specifications}
\begin{itemize}
    \item 8 cores CPU, 4.8 Ghz sustained clock while boosting
    \item 32 GB of 3600 Mhz 
    \item SSD $\approx$ 3000 MB/s in read and write
\end{itemize}
\section*{Contributions}
In this first project proposal part we more or less worked together, most of the time was spent doing organization, sharing ideas and compiling this report so the contribution of each member was approximately 1/3.
\begin{description}[font=\normalfont\itshape]
    \item[Lavorati Ippolito:] Helped writing the report, helped finding the datasets and assess their quality, research and presented ideas, some of which were discarded. Researched statistical tests and how to apply them
    \item[Orsolon Ludovico:] Helped writing the report, gave the main idea about the friendship paradox, formulated of the hypothesis and researched about experiments to test it. Wrote part of the code
    \item[Stefani Patrizia:] Helped writing the report, helped finding the datasets, research on the methods to implement the experiments and relative python functions, implemented part of the methods for the computation of graph statistics
\end{description}
\begin{thebibliography}{9}
\bibitem{friendship_paradox}
Scott L. Feld (1991) \emph{Why Your Friends Have More Friends
than You Do}, The University of Chicago.
\end{thebibliography}
\end{document}

