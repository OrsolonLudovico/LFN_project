\documentclass{article}
\usepackage{graphicx}
\PassOptionsToPackage{hyphens}{url}\usepackage{hyperref}
\usepackage{titling}
\setlength{\droptitle}{-13em}
\usepackage{enumitem}
\setlist[description]{leftmargin = 0mm}
\usepackage{amsmath}
\usepackage{amsfonts}
\usepackage{float}
\usepackage{makecell}

\title{An analysis of the \textit{friendship paradox} on different types of social networks}
\author{Ippolito Lavorati, Ludovico Orsolon, Patrizia Stefani}
\date{November 2024 - January 2025}


\begin{document}
\maketitle
\section{Motivations}
In recent decades, social media have become increasingly popular, transforming the way people connect and interact; examining these platforms from a graph network perspective can thus provide valuable insights into the social dynamics they enclose.\\
One particularly interesting phenomenon that we decided to explore is the \textit{friendship paradox}, a concept formulated by sociologist Scott L. Feld, which suggests that, "on average, an individual's friends have more friends than the individual does." \cite{friendship_paradox}.
This analysis will investigate how each considered platform reflects this paradox, aiming to understand the extent to which the phenomenon appears across these networks.

\subsection{Datasets}
We’ve chosen three graphs, each one representing the friendship in a different social network:
\begin{itemize}
    \item Youtube: the dataset at \url{https://snap.stanford.edu/data/com-Youtube.html} contains information about the networks of followers on Youtube communities 
    \item Linkedin: the dataset at \url{https://networkrepository.com/soc-linkedin.php} contains the user-to-user connections
    \item Facebook: the dataset at \url{https://networkrepository.com/socfb-wosn-friends.php} contains a network of users' friendships
\end{itemize}
We think that these three social networks are different enough in nature to have different characteristics between each other; for example, we expect that a 'popular user' on youtube will have many more 'friends' than the average one, while 'popular users' on Facebook will have fewer friends in proportion due to the fact that often unpopular users are still friends with a fair number of people (such as friends and family in real life).\\
Furthermore, since the datasets utilized are huge, we designed a smaller graph to debug our program and verify our measures (Figure~\ref{fig:testgraph}). We called this graph \textit{TestGraph}.
\begin{figure}[H]
    \centering
    \includegraphics[width=0.8\textwidth]{TestGraph.png}
    \caption{\textit{TestGraph}}
    \label{fig:testgraph}
\end{figure}
We created a random graph for each dataset in order to better verify the statistical analysis. The random graphs are constructed with the same number of nodes and edges of the real graph utilized using the \textit{gnm\_random\_graph} function from the \textit{NetworkX} library.\\
Generating a single random graph for each social network is certainly not optimal; we would need many random graphs to better estimate random features. We chose not to do so for the sake of computational time: computing even a single measure on our machines was a long process, so we decided to use our time trying to understand results rather than computing tons of measures.\\
The graphs analyzed, as well as the respective random graphs are characterized by the following properties:
\begin{table}[ht]
\centering
\begin{tabular}{|c|c|c|c|c|}
\hline
& TestGraph & Facebook & Linkedin & Youtube \\ \hline
Nodes & $22$ & $63\,731$ & $6\,726\,011$ & $1\,134\,890$\\ \hline
Edges & $34$ & $817\,090$ & $19\,360\,690$ & $2\,987\,624$ \\ \hline
Avg Degree & 3.09 & 25.64 &  5.76 & 5.26 \\ \hline 
\end{tabular}
\caption{Some graphs' properties}
\end{table} 

\section{Methods}
The program responsible for the analysis is written in \textit{Python} using the \textit{NetworkX} library for graphs, the \textit{tqdm} library to keep track of long processes through progress bars, \textit{pyplot} from \textit{matplotlib} to visualize information, \textit{pandas} to deal with csv files. The files containing edge information, downloaded from the datasets, will serve as input to our programs. Some statistical analysis were also performed in R.
The program was developed collaboratively on GitHub\cite{Github_repo} and all computations provided in this paper can be found in one of the functions provided on it.

\subsection{The fship\_score}
We decided to encapsulate the concept of the friendship score in a measure that we called \textit{fship\_score}, computed for each node. 
The \textit{fship\_score} is defined as:
\\\[
\textit{\textbf{fship\_score(u)}} = \deg(u) \cdot \left( \frac{\sum_{v \in N(u)} \deg(v)}{\deg(u)} \right)^{-1} =
\frac{\deg(u)^{2}}{\sum_{v \in N(u)} \deg(v)}
\]\\
That is, the degree of the considered node divided by the average degree of its neighbors. Intuitively if the score is less or equal to 1 the friendship paradox is true for the node, otherwise we consider it an outlier.


\section{Experiments}
\subsection{Verifying the \textit{friendship paradox}}
First of all we need to check if the \textit{friendship paradox} is present in our datasets: below a table with the proportion of outliers is reported (\textit{number of nodes with $fship\_score > 1$ $/$ $total\_number\_of\_nodes$}), this table contains results from one of our R scripts. 
\begin{table}[ht]
\centering
\begin{tabular}{|c|c|c|c|c|}
\hline
& TestGraph & Facebook & Linkedin & Youtube \\ \hline
Real & 0.2777 & 0.0865 & 0.2927 & 0.0325\\ \hline
Random & 0.3636 & 0.4086 & 0.3556 & 0.3544 \\ \hline
\end{tabular}
\caption{Proportion of outliers}
\end{table} 
\\

We can immediately see that the \textit{friendship paradox} is true for both real and random graphs but we can also see that it is more accentuated on the real graphs than on the random ones: this suggests that the \textit{friendship paradox} is a characteristic of real "social network" graphs even though it can be also present in random graphs to some degree. Furthermore, we can see that the paradox is more accentuated on Youtube and less on Linkedin as expected. From the data, the random graphs seem to have a different proportion of outliers than the real graphs (H1), to verify this hypothesis we performed the z-test on the values in the above table using [real graph - random corresponding graph] pairs.
\[z = \frac{p_1 - p_2}{\sqrt{p_{comb} (1 - p_{comb}) \left( \frac{1}{n_1} + \frac{1}{n_2} \right)}}\]
where:
\[n_i = \textit{number of nodes in graph i}\]
\[k_i = \textit{number of nodes with fship\_score $>$ 1 in graph i}\]
\[p_i = \frac{k_i}{n_i}\]
\[p_{comb} = \frac{k_1 + k_2}{n_1 + n_2}\]


These are the resulting p-values: \\
\begin{table}[ht]
\centering
\begin{tabular}{|c|c|c|c|c|}
\hline
& TestGraph & Facebook & Linkedin & Youtube \\ \hline
Real vs Random & 0.5 & 0 & 0 & 0\\ \hline
\end{tabular}
\caption{p-values Real vs Random z-test (computed by our R function)}
\end{table} 
\\
The p-values are all zero (reject \(H_0\)) except for the artificial graph suggesting that the proportion of outliers are different between the random and the real graphs, this means that the \textit{fship\_score} is probably not due to randomness. We also compared datasets against each other to see if there are common proportions between them, but the p-values are zero for all combinations (facebook-Linkedin, Linkedin-Youtube and Youtube-Facebook) so we conclude that such correlation doesn't exists.

\subsection{\textit{Friendship paradox} correlations}

It's important to check if the \textit{friendship paradox} is related to some other metric of the graph. We tried to analyze three key metrics: \textit{degree}, \textit{PageRank}, and \textit{clustering coefficient} using:
\begin{itemize}
    %\item \textit{Linear regression} provides a clear model of how the friendship paradox score varies with each metric, offering insights into the strength and direction of these associations.
    \item \textit{Pearson correlation} measures the degree of linear relationships
    \item \textit{Spearman correlation} assesses monotonic associations, uncovering potential nonlinear trends.
\end{itemize}
We also tried to test \textit{closeness} and \textit{betweenness centrality} but computing these metrics (even approximated) for large-scale networks such as Facebook, Linkedin, and Youtube requires too much computational power, not available to us; we hence marked them as unfeasible. \\
Additionally, although linear regression is a common method for modeling relationships between variables, it was not applied in this study.
Instead, we relied on Pearson and Spearman correlations, which have proven to be more effective in quantifying and understanding the associations between the friendship paradox and the selected metrics. These methods have revealed the strength and nature of associations while ensuring statistical significance. \\
The results presented in this section were obtained using the code implemented in the file \textit{analysis.py}; they are shown in the following table: \\
\begin{table}[H]
    \centering
    \begin{tabular}{|c|c|c|c|c|c|c|}
        \hline
        & \makecell{Facebook \\ Pearson} & \makecell{Facebook \\ Spearman} & \makecell{Linkedin \\ Pearson}
         & \makecell{Linkedin \\ Spearman} & \makecell{Youtube \\ Pearson} & \makecell{Youtube \\ Spearman} \\ \hline
        \textbf{Degree} & 0.784 & 0.747   & 0.728             & 0.865           & 0.575           & 0.185       \\  \hline
        %Pearson & 0.784 & 0.728 & 0.575 \\ \hline
        %Spearman  & 0.747 & 0.865 & 0.185 \\ \hline

        \textbf{PageRank} & 0.898 & 0.906 & 0.967 & 0.980 & 0.836 & 0.743 \\ \hline
        %Pearson & 0.898 & 0.967 & 0.836 \\ \hline
        %Spearman  & 0.906 & 0.980 & 0.743 \\ \hline

        \textbf{Clustering} &  -0.106 &  0.111 & -0.355 & -0.230 & -0.003 & 0.064\\ \hline
        %Pearson & -0.106 & -0.355 & -0.003 \\ \hline
        %Spearman  & 0.111 & -0.230 & 0.064 \\ \hline
    \end{tabular}
    \caption{Metrics correlation}
\end{table} 

Each value represents the correlation coefficient between the \textit{fship\_score} and the corresponding metric. Values closer to 1 or -1 indicate stronger correlations, with positive values showing direct relationships and negative values indicating inverse relationships.
For instance, in \textit{Facebook}, PageRank has the strongest positive correlation with the \textit{fship\_score} (Pearson = 0.898, Spearman = 0.906), highlighting the influence of hubs in the network. Conversely, the clustering coefficient generally shows weak or negligible correlations, emphasizing its limited impact.

In \textbf{Facebook}, \textit{PageRank} showed the strongest correlation with the \textit{friendship paradox}, underscoring the influence of highly connected hubs. \textit{Degree} also showed a significant positive correlation, reinforcing its central role in the paradox, while clustering coefficient had minimal negative correlation, suggesting that local cohesion has little impact. \\
For \textbf{Linkedin}, \textit{PageRank} again emerged as the most significant factor, with stronger correlations than those observed in Facebook. \textit{Degree} correlations were also substantial, reflecting Linkedin's emphasis on direct professional connections. The clustering coefficient showed a slightly more negative correlation, but remained a secondary factor. \\ 
Finally, in \textbf{Youtube}, \textit{PageRank} remained a strong driver of the \textit{friendship paradox}, although its correlation was weaker than Facebook and Linkedin. \textit{Degree} correlations were less pronounced, indicating different dynamics shaped by content creation and consumption. The clustering coefficient showed an almost negligible correlation, underscoring its limited relevance in this network. \\
These results highlight the multifaceted nature of the \textit{friendship paradox}, driven primarily by global metrics such as PageRank and degree, while local metrics such as clustering have a more subtle impact.

\section{Conclusions}
From our analysis we can conclude that the \textit{friendship paradox} exists and it's manifested in social network communities; we also conclude that the \textit{PageRank centrality} is closely related to the \textit{friendship paradox} since it seems to be a good indicator for the \textit{fship\_score}. We can hypothesize that this correlation is due to the fact that both \textit{PageRank} and \textit{fship\_score} are measures of the "popularity of a node", in fact \textit{PageRank} was created aiming to distinguish popular websites and put them at the top of a web search. This correlation seems worth to be analyzed further in the future. \\
For the \textit{degree} we have a similar situation: there is surely a correlation between \textit{degree} and \textit{fship\_score} but there is also a correlation between \textit{degree} and \textit{PageRank centrality}. What's happening is that probably the correlation we see with the \textit{degree} is a byproduct of the correlation that exists with \textit{PageRank centrality} rendering the \textit{degree} correlation less interesting overall.\\
Finally, the \textit{clustering coefficient} exhibited weak or negligible correlations, suggesting that local cohesiveness has minimal influence on the phenomenon. 
This reinforces the idea that global metrics like \textit{PageRank} are the primary drivers of the \textit{friendship paradox} hence we suspect that other centrality measures will be related to it as well.

\section{Possible improvements}
With more powerful computational machines and more time there are a few ideas that could be developed to expand on this paper results:
\begin{itemize}
    \item Generating many random graphs for each social network would provide a better approximation of the randomly-generated features and also, generating other types of random graphs (Chung-Lu model for example), could highlight new correlations we may have missed.
    \item An obvious improvement could be simply using more data, for example finding new graphs from other social networks, we could also check if in different graphs obtained from the same social network we find significative differences on \textit{fship\_score}.
    \item Inclusion of Closeness and Betweenness Centrality: these metrics, while computationally expensive for large-scale networks, could provide valuable insights into the global and intermediary roles of nodes in the friendship paradox. Future work could focus on optimizing their computation to integrate them into the analysis effectively.
\end{itemize}

\section{Machine specifications}
\begin{itemize}
    \item 8 cores CPU, 4.8 Ghz sustained clock while boosting
    \item 32 GB of 3600 Mhz 
    \item SSD $\approx$ 3000 MB/s in read and write
\end{itemize}
\section*{Contributions}
We worked together and believe that the the difference in contribution amount between all of us was negligible, hence we report a contribution of 1/3 for each member.
\begin{description}[font=\normalfont\itshape]
    \item[Lavorati Ippolito:] Report, helped finding the datasets, researched and presented ideas, statistical tests and interpretation, python programming
    \item[Orsolon Ludovico:] Report, main idea about the friendship paradox, formulated of the hypothesis, researched about experiments, python programming, statistical tests and interpretation
    \item[Stefani Patrizia:] Report, found datasets, researched on the methods to implement the experiments and relative python functions, implemented some methods for the computation of graph statistics, implemented the configuration logic
\end{description}
\begin{thebibliography}{9}
\bibitem{friendship_paradox}
Scott L. Feld (1991) \emph{Why Your Friends Have More Friends
than You Do}, The University of Chicago.
\bibitem{Github_repo}
The program can be found on GitHub at \url{https://github.com/OrsolonLudovico/LFN_project.git}
\end{thebibliography}
\end{document}